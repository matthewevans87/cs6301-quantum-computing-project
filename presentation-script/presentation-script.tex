\documentclass[11pt]{article}
\usepackage[T1]{fontenc}
\usepackage[utf8]{inputenc}
\usepackage[margin=1in]{geometry}
\usepackage{setspace}
\onehalfspacing
\usepackage{titlesec}
\usepackage{amsmath}
\usepackage{hyperref}
\hypersetup{
    colorlinks   = true,
    linkcolor    = blue,
    urlcolor     = blue
}

\titleformat{\section}{\normalfont\Large\bfseries}{\thesection.}{1em}{}

\title{Presentation Script \\ 
\Large{Response to Ramesh \& Vinay, (2003)\\ 
\small{\textit{String Matching in \(\tilde{O}(\sqrt{n} + \sqrt{m})\) Quantum Time}} }}

\author{%
\normalsize{Matthew Evans, Ariz Siddiqui, Nathan Puskuri}
}
\date{\today}
\begin{document}
\maketitle

\section*{Order of Speakers}
\begin{enumerate}
  \item Matt
        \begin{itemize}
          \item Introduce team
          \item Intro Slide
          \item Conceptual Dependencies Figure
        \end{itemize}
  \item Nathan \begin{itemize}
          \item Grover's Algorithm
          \item BBHT96
          \item Probablistic Oracles
          \item String Matching in \(O(\sqrt{n} \sqrt{m})\)
        \end{itemize}
  \item Matt \begin{itemize}
          \item Deterministic Sampling
          \item DH96: Minimum Finding Oracle
          \item The Algorithm (steps)
        \end{itemize}
  \item Ariz \begin{itemize}
          \item Periodicity
          \item Constant Two-sided failure probability
          \item Achieving \(\tilde{O}(\sqrt{n} + \sqrt{m})\)
          \item Conclusion
        \end{itemize}
\end{enumerate}

\section*{Introduction}
\textbf{Matt} \linebreak
Hello everyone, my name is Matt and these are my teammates Nathan, and Ariz.
We will be presenting on paper by \textit{Ramesh and Vinay} in which they provide a quantum string matching algorithm. The problem can be stated as follows: Given a text $t$ of length $n$ and a pattern $p$ of length $m$, decide whether $p$ occurs in $t$. The classical approach to this problem requires $\Theta(n + m)$.

The authors which exploit amplitude amplification to beat this linear time and propose a quantum string-matching algorithm running in \(\widetilde{O}\bigl(\sqrt{n} + \sqrt{m}\bigr)\) with two-sided error probability.

\section*{Conceptual Dependencies Figure}
\textbf{Matt} \linebreak
Here we provide a figure mapping the conceptual dependencies the authors leverage to achieve their result.
Grover's algorithm is central to the author's algorithm. On the left, we see the result of the BBHT96 paper, which provides a rigorous complexity bound on Grover's. On the right we see the need for a so-called "Minimum Finding Oracle" by Durr and Høyer which enables finding a ``minimum element''. Deterministic Sampling is a procedure introduced by Vishkin for reducing the number of comparisons needed between a pattern and text to identify matches. The authors considers two scenarios: matching aperiodic patterns, and matching those \textit{with} periodicity.
All of these building blocks come together to produce the author's quantum string matching algorithm with time complexity \(\widetilde{O}(\sqrt{m} + \sqrt{n})\).

\section*{Grover's Algorithm}
\textbf{Nathan} \linebreak
\section*{BBHT96}
\textbf{Nathan} \linebreak
\section*{Probablistic Oracles}
\textbf{Nathan} \linebreak
\section*{String Matching in \(O(\sqrt{n} \sqrt{m})\)}
\textbf{Nathan} \linebreak

% 1:05
\section*{Deterministic Sampling}
\textbf{Matt}\linebreak

Imagine matching a pattern of length \(m\) against a sub-block of text of length \(\lceil m/2\rceil\). Naively, there are \(m/2\) possible alignments between the pattern and the text within this block, each costing \(\widetilde{O}(\sqrt{m})\) quantum time to check. Deterministic sampling is a procedure that selects an \(O(\log m)\)-sized set of pattern indices which rule out all but one of these alignments.

The Key idea is that among the \(m/2\) alignments of the pattern with the text block, aperiodicity guarantees there to be a pattern index where at least two copies differ. Hence, checking for a pattern match at each index in \(S\) eliminates at least half the candidates. Repeating this halving process \(O(\log m)\) times results in a single surviving alignment.

Classically each round would compare \(O(m)\) characters. Quantum search and \textit{minimum-finding} reduce this to \(\widetilde O(\sqrt{m})\), which is the source of the \(\widetilde O(\sqrt{m})\) term in the algorithm's overall \(\widetilde O(\sqrt{n} + \sqrt{m})\) time.


% 0:50
\section*{DH96 - Minimum Finding Oracle}
\textbf{Matt}\linebreak
As mentioned, the algorithm uses a ``minimum-finding'' procedure, specifically, a minimum-finding oracle, leveraged by a Grover's-style search.

In minimum finding, one begins by selecting a random index \(k\) and then uses Grover’s search to locate any index \(i\) such that \(\text{item}[i]\) is less than \(\text{item}[k]\). If none is found, \(k\) is the minimum; otherwise, one updates \(k\) to the value of \(i\) and repeats the search a constant number of times. Each quantum search costs \(\widetilde{O}(\sqrt{n/t})\), where \(t\) is the current number of smaller elements. The geometric reduction of \(t\) yields an overall \(\widetilde{O}(\sqrt{n})\) runtime. This procedure transforms classical linear‐time minimum finding into a square‐root‐time subroutine, which is used to identify the leftmost text block that may containing a match.

% 1:10
\section*{The Algorithm}
\textbf{Matt} \linebreak

Now let's see how these pieces fit together.

On the left we see the set \(S\) of sample indices being returned from Deterministic Sampling, and the ``database'' \(D_B\) containing the \(B\) partitions of the text, each with length \(m/2\).

In the central column, we have an alternating pattern of Oracle construction and applications of Grover's Algorithm.

First, the so-called ``hit'' oracle \(h\) is constructed based on the set \(S\).
An application of Grover's uses oracle \(h\) over database \(D_B\) to find the index of a candidate block, \(i^\star\).

Second, the ``candidate'' oracle \(k\) is constructed based on index \(i^\star\). Grover's is applied using oracle \(k\) to the database \(D_J\) containing all possible alignments within block \(i^\star\). This procedure finds a the sub-index \(j^\star\) where the indices in \(S\) aligns with the text.

Lastly, the ``mismatch'' oracle \(g\), which is an application of the ``minimum finding oracle'', is constructed based on \(i^\star\) and \(j^\star\). Grover's is applied a third time, and performs a full verification over all indices \(\ell\) within the selected alignment \(j^\star\) in block \(i^\star\).

If no mismatches are found, then we know the given pattern matches the text, specifically, at the index \(i^\star + j^\star\).



% TODO: ensure none of this is critical
% \section*{Wrap-up}
% The DH96 minimum-finding oracle turns a linear scan into an \(O(\sqrt{n})\) quantum subroutine. By interleaving random picks with Grover searches for improvements, it guarantees a high-probability success in square-root time. This powerful tool underlies every step in the paper where we need the “smallest” or “first” index—crucial for block selection, sample building, and periodic-pattern handling.

\section*{Periodicity}
\textbf{Ariz} \linebreak
\section*{Constant Two-sided failure probability}
\textbf{Ariz} \linebreak
\section*{Achieving \(\tilde{O}(\sqrt{n} + \sqrt{m})\)}
\textbf{Ariz} \linebreak
\section*{Conclusion}
\textbf{Ariz} \linebreak

\end{document}