\documentclass[12pt]{article}
% \documentclass[12pt]{IEEEtran}

\usepackage[utf8]{inputenc}
\usepackage{amsfonts,amsmath,amsthm,amssymb,dsfont,mathtools}
\usepackage{geometry}
\usepackage{graphicx}
\usepackage{hyperref}
\usepackage{enumitem}
\usepackage{bookmark}
\usepackage{cite}
\usepackage{algorithm}
\usepackage{algpseudocode}
\usepackage{braket}
\usepackage{tikz}
\newcommand{\ketbra}[2]{\mathinner{|{#1}\rangle \langle{#2}|}}
\newcommand*\xor{\oplus}
\bibdata{references}
\usetikzlibrary{quantikz2}

\geometry{a4paper, margin=1in}


\title{\Large{Response to Ramesh \& Vinay, (2003)\\ \small{\textit{String Matching in \(\tilde{O}(\sqrt{n} + \sqrt{m})\) Quantum Time}} }}

\author{%
\normalsize{Matthew Evans, Ariz Siddiqui, Nathan Puskuri}
}
\date{\today}

\begin{document}
\maketitle

\section{Citation Example}
This is a dummy citation \cite{RameshH2003SmiO}.

\section{Matrix and Align Examples}
\begin{align*}
    H & = \frac{1}{\sqrt{2}}\begin{bmatrix} 1 & 1 \\ 1 & -1 \end{bmatrix} \\
    X & = \begin{bmatrix} 0 & 1 \\ 1 & 0 \end{bmatrix}                    \\
    Y & = \begin{bmatrix} 0 & -i \\ i & 0 \end{bmatrix}                   \\
    Z & = \begin{bmatrix} 1 & 0 \\ 0 & -1 \end{bmatrix}
\end{align*}

\[
    \begin{bmatrix} 1 & 0 & 0 & 0 \\ 0 & 1 & 0 & 0 \\ 0 & 0 & 1 & 0 \\ 0 & 0 & 0 & -1 \end{bmatrix}
\]

\section{Quantikz examples}
\begin{center}
    \begin{quantikz}
        & & \\
        & \gate{H}&
    \end{quantikz}
\end{center}

\begin{figure}[ht]
    \resizebox{.5\textwidth}{!}{
        \centering
        \begin{quantikz}
            &         &        &                &\ctrl{2}&        &        &                & \ctrl{2}&                &\ctrl{1}&                &\ctrl{1}&\gate{T}&\\
            &         &\ctrl{1}&                &        &        &\ctrl{1}&                &         &\gate{T^\dagger}&\targ{} &\gate{T^\dagger}&\targ{} &\gate{S}&\\
            &\gate{H} &\targ{} &\gate{T^\dagger}&\targ{} &\gate{T}&\targ{}  &\gate{T^\dagger}&\targ{} &\gate{T}        &\gate{H}                 &        &        & &
        \end{quantikz}
    }
\end{figure}

\begin{quantikz}
    &\gate{X} & \ctrl{1} & \\
    & & \targ{} &
\end{quantikz}

\begin{quantikz}
    & \ctrl{1} & \gate{X} & \\
    & \targ{} & \gate{X} &
\end{quantikz}

\begin{quantikz}
    & \ctrl{1} & \\
    & \gate{Z}&
\end{quantikz}

\begin{quantikz}
    & \gate{Z} & \\
    & \ctrl{-1}&
\end{quantikz}

\section{Bra-Ket examples}

\[
    \begin{aligned}
        \ket{\Psi} & = \frac{1}{\sqrt{2}}\Bigl[H\ket{0}(\alpha \ket{\phi_+} + \beta \ket{\phi_-})
        + H\ket{1}(\alpha \ket{\phi_+} - \beta \ket{\phi_-})\Bigr]                                                                                 \\
                   & =\frac{1}{2}\Bigl\{\ket{0}\Bigl[(\alpha \ket{\phi_+} + \beta \ket{\phi_-})+ (\alpha \ket{\phi_+} - \beta \ket{\phi_-})\Bigr]  \\
                   & \quad\quad + \ket{1}\Bigl[(\alpha \ket{\phi_+} + \beta \ket{\phi_-})- (\alpha \ket{\phi_+} - \beta \ket{\phi_-})\Bigr]\Bigr\} \\
                   & =\alpha \ket{0}\ket{\phi_+} + \beta \ket{1}\ket{\phi_-}.
    \end{aligned}
\]

\begin{align*}
    \ket{\psi_1} = \ket{0} & : \ket{\psi_2} \rightarrow \ket{\psi_2}   \\
    \ket{\psi_1} = \ket{1} & : \ket{\psi_2} \rightarrow Z\ket{\psi_2}.
\end{align*}


\bibliographystyle{plain}
\bibliography{references}


\end{document}
